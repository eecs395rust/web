% vim: ts=2
\documentclass{beamer}

\usepackage{solarslides}
\usepackage{graphicx}
\usepackage{tikz}

\begin{document}

\begin{frame}
\thispagestyle{empty}\centering
\ti{\alert{$n$-way Mutual Exclusion}}
\hd{EECS 3/495 “Rust”}
\hd{Spring 2017}
\end{frame}

\begin{frame}[fragile]{Filter algorithm for $n$ threads}{}
  ««
  |K‹template› <|T‹int› N>
  |K‹class› |D‹Filter_lock› : |K‹public› |D‹Lock_base›
  {
  	|+|kill%
    |T‹int› level_|T‹[N]› = {};
    |T‹int› waiting_|T‹[N]›;

    |T:bool exists_competition(|T:int level)
    {
    	|+|kill%
      |K:for (|T:auto k : |D‹thread›::all_ids())
      	|K:if (k != i() |K‹&&› level_[k] >= level)
      		|K‹return› |L‹true›;
      |K:return |L‹false›; |-
    }
    $|vdots$ |-
  }
  »»
\end{frame}

\begin{frame}[fragile]{}{}
  ««
  |K‹template› <|T‹int› N>
  |K‹class› |D‹Filter_lock› : |K‹public› |D‹Lock_base›
  {
  $|vdots$
  |K‹public›:
  	|+|kill%
    |K‹virtual› |T‹void› lock() |Q‹override›
    {
    	|+|kill%
      |K‹for› (|T:int level = |L1; level < N; ++level) {
      	|+|kill%
        waiting_[level] |==|kill%
        level_[i()] |> = level;
        waiting_[level] = i();
        |K:while (|=exists_competition(level) |K‹&&›
                  |>waiting_[level] == i()) {} |-
      } |-
    }
    
    |K‹virtual› |T:void unlock() |Q‹override›
    { level_[i()] = |L0; } |-
  }
  »»
\end{frame}

\begin{frame}{Filter lock properties}
  \begin{itemize}
    \item Mutual exclusion
      \begin{itemize}
        \item By induction, one thread gets stuck in each level...
      \end{itemize}
    \item Deadlock freedom
      \begin{itemize}
        \item Like Peterson—only one thread can wait per level
      \end{itemize}
    \item Starvation freedom
      \begin{itemize}
        \item Like Peterson—every thread advances if any does
      \end{itemize}
    \pause
  \item Fairness?
    \begin{itemize}
      \item No—threads can overtake others
    \end{itemize}
  \end{itemize}
\end{frame}

\begin{frame}{Bounded waiting}{}
  Idea: If thread $A$ ``starts before'' $B$, then $A$ enters CS before
  $B$.

  \pause
  But what is “starts before”?
\end{frame}

\begin{frame}{Bounded waiting}{}
  Divide \emph{lock()} operation into two intervals:
  \begin{itemize}
    \item Doorway ($D_A$), finite steps
    \item Waiting ($W_A$), possibly unbounded
  \end{itemize}

  \pause
  $r$-Bounded Waiting Guarantee: If $D_A^k \to D_B^j$, then $CS_A^k \to
  CS_B^{j + r}$.

  \pause
  “If $A$ enters the doorway for the $k$th time before $B$ enters it for the
  $j$th time, then $A$’s $k$th critical section happens before $B$’s $(j
  + r)$th critical section.”
\end{frame}

\begin{frame}{$r$-Bounded waiting}
  \begin{itemize}
    \item Peterson's Algorithm (for 2) has $r = 0$
      (first-come-first-served)
    \item Filter algorithm (for $n$) has $r = \infty$
      \pause
    \item Bakery algorithm (for $n$) has has $r = 0$ (first-come-first-served)
  \end{itemize}
\end{frame}

\begin{frame}[fragile]{Helper class: lexicographically-ordered pairs}{}
  ««
  |K:template <|K:typename |T A, |K:typename |T B>
  |K:struct |D‹LP›
  {
  	|T A x;
  	|T B y;
  };

  |K:template <|K:typename |T A, |K:typename |T B>
  |T:bool operator>(|=|T‹const LP<A, B>› |K‹&› p,
                    |>|T‹const LP<A, B>› |K‹&› q)
  {
  	|K:return p.x > q.x |K‹||› (p.x == q.x |K‹&&› p.y > q.y);
  }
  »»
\end{frame}

\begin{frame}[fragile]{Bakery algorithm for $n$ threads}{}
  ««
  |K‹template› <|T‹int› N>
  |K‹class› |D‹Bakery_lock› : |K‹public› |D‹Lock_base›
  {
  	|+|T‹int› max_label_ |==|kill%
    |T‹bool› flag_|T‹[N]› |>= {|L‹false›};
    |T‹int› label_|T‹[N]› |>= {|L0};
    |T‹int› max_label_    |>= |L0;

    |T:bool someone_is_ahead()
    {
    	|+|kill%
      |K:for (|T:auto k : |D‹thread›::all_ids())
      	|K:if (flag_[k] |K‹&&› |D‹LP›{label_[i()], i()} > |D‹LP›{label_[k], k})
      		|K‹return› |L‹true›;
      |K:return |L‹false›; |-
    }
    $|vdots$ |-
  }
  »»
\end{frame}

\begin{frame}[fragile]{}{}
  ««
  |K‹template› <|T‹int› N>
  |K‹class› |D‹Bakery_lock› : |K‹public› |D‹Lock_base›
  {
  $|vdots$
  |K‹public›:
  	|+|kill%
    |K‹virtual› |T‹void› lock() |Q‹override›
    {
    	|+|kill%
      label_[i()] |==|kill%
      flag_[i()]  |>= |L‹true›;
      label_[i()] |>= ++max_label_;
      |K:while (someone_is_ahead()) {} |-
    }
    
    |K‹virtual› |T:void unlock() |Q‹override›
    { flag_[i()] = |L‹false›; } |-
  }
  »»
\end{frame}

\begin{frame}{Bakery Y2$^{\text{32}}$K bug}{}
  Does overflow matter?

  \centering
  \begin{tabular}{|c|c|}
    \hline
    \textbf{Bits} & \textbf{Does it?} \\
    \hline
    \hline
    16 & quite \\
    32 & maybe \\
    64 & no \\
    \hline
  \end{tabular}
\end{frame}

\begin{frame}{Bakery lock properties}
  \begin{itemize}
    \item Mutual exclusion
      \begin{itemize}
        \item Between any two (label[k], k) pairs, one will defer to the
          other\ldots
      \end{itemize}
    \item Deadlock freedom
      \begin{itemize}
        \item Must be some least (label[k], k) pair
      \end{itemize}
    \item Starvation freedom
      \begin{itemize}
        \item No thread takes the same number twice
      \end{itemize}
    \item First-come-first-served (0-bounded waiting)
      \begin{itemize}
        \item First through the door has lower label, goes first
      \end{itemize}
      \pause
    \item \textbf{Practical?}
      \begin{itemize}
        \item Have to readh $n$ variables to lock
      \end{itemize}
  \end{itemize}
\end{frame}

\begin{frame}{“Registers” (shared memory locations)}{}
  Flavors:
  \begin{itemize}
    \item Multi-reader/single-writer (flag[])
    \item Multi-reader/multi-writer (waiting)
    \item (Not that interesting: SRMW and SRSW)
  \end{itemize}
\end{frame}

\begin{frame}{Theorem}{}
  At least $n$ MR\alt<3->MSW
  (multi-reader/\alt<2->{\alert<2-3>{multi}}{single}-writer)
  registers are needed to
  solve deadlock-free mutual exclusion.

  \uncover<4>{
    \textbf{Proof sketch.} For $n$ = 2, one register is insufficient
    because neither thread necessarily sees the other’s write. Then by
    induction, the record of the first thread to enter always gets
    obliterated by the rest.
  }
\end{frame}

\begin{frame}{Summary}{}
  For deadlock-free mutual exclusion of $n$ threads:
  \begin{itemize}
    \item Best known algorithm uses $2n$ MRSW registers
    \item Lower bound for MRMW is $n$
  \end{itemize}

  \pause
  $O(n)$ reads is too inefficient—we need something better, and we’ll get
  it from the hardware
\end{frame}

\begin{frame}{}{}
\vskip4pt
\parskip=4pt
\scriptsize
This work is licensed under a Creative Commons “Attribution-ShareAlike
3.0 Unported” license.

These slides are derived from the companion slides for \emph{The Art of
Multiprocessor Programming,} by Maurice Herlihy and Nir Shavit. Its
original license reads:
\begin{quote}
  This work is licensed under a \emph{Creative Commons Attribution-ShareAlike
  2.5 License.}
\begin{itemize}
  \item \textbf{You are free}:
    \begin{itemize}
      \item\tiny \textbf{to Share} — to copy, distribute and transmit the work
        \item\tiny \textbf{to Remix} — to adapt the work
    \end{itemize}
  \item \textbf{Under the following conditions:}
    \begin{itemize}
      \item\tiny \textbf{Attribution.} You must attribute the work to “The Art of
        Multiprocessor Programming” (but not in any way that suggests
        that the authors of that work or this endorse you or your use of
        the work).
      \item\tiny \textbf{Share Alike.} If you alter, transform, or build upon this work,
        you may distribute the resulting work only under the same,
        similar or a compatible license.
    \end{itemize}
  \item For any reuse or distribution, you must make clear to others the
    license terms of this work. The best way to do this is with a link to
    \begin{itemize}
      \item\tiny \url{http://creativecommons.org/licenses/by-sa/3.0/}.
    \end{itemize}
  \item Any of the above conditions can be waived if you get permission from
    the copyright holder.
  \item Nothing in this license impairs or restricts the author’s moral
    rights.
\end{itemize}
\end{quote}
\end{frame}

\end{document}
