% vim: ts=2
\documentclass{beamer}

\usepackage{solarslides}
\usepackage{graphicx}
\usepackage{tikz}

\begin{document}

\begin{frame}
\thispagestyle{empty}\centering
\ti{\alert{Concurrent Objects and Linearizability}}
\hd{EECS 3/495 “Rust”}
\hd{Spring 2017}
\end{frame}

\begin{frame}{What is a concurrent object?}{}
  \begin{itemize}
    \item<1-2> How do we \alert{describe} one?
    \item<1> How do we \alert{implement} one?
    \item<1-2> How do we \alert{tell if we’re right}?
  \end{itemize}
\end{frame}

\begin{frame}[fragile]{Case study: FIFO queue}{}
  \begin{center}
    q = \begin{tabular}{|c|c|c|c}
      \hline
      \uncover<-4>{{\textL2}} & {\textL4} & \uncover<3->{{\textL6}} & \\
      \hline
    \end{tabular}
  \end{center}
  \uncover<2->{«q.enq({\textL 6})»}

  \uncover<4->{«q.deq()»} \uncover<5->{$\Rightarrow$ \textL2}
\end{frame}

\begin{frame}[fragile]{Implementation: Lock-based ring buffer}{}
  ««
  |P‹|#include <array>›

  |K:template <|K:typename Element, |T:int capacity>
  |K:class |D‹Lock_based_FIFO›
  {
  |K‹public›:
  	|+|T:void enq(|T‹Element›);
    |T:Element deq(); |-

  |K‹private›:
  	|+|T‹std::array<Element, capacity>› data_;
    |T:unsigned head_ = |L0, tail_ = |L0;
    |T:Lock lock_; |-
  };
  »»
\end{frame}

\begin{frame}[fragile]{Implementation: Lock-based enqueue}{}
  ««
  |K:template <|K:typename Element, |T:int capacity>
  |T:void |D‹Lock_based_FIFO<Element, capacity>›::enq(|T:Element x)
  {
  	|+|T:LockGuard guard(lock_);

    |K:if (tail_ - head_ == capacity) |K:throw |D‹fifo_full›();

    data_[tail_++ |% capacity] = x; |-
  }
  »»
\end{frame}

\begin{frame}[fragile]{Implementation: Lock-based dequeue}{}
  ««
  |K:template <|K:typename Element, |T:int capacity>
  |T:Element |D‹Lock_based_FIFO<Element, capacity>›::deq()
  {
  	|+|T:LockGuard guard(lock_);

    |K:if (tail_ == head_) |K:throw |D‹fifo_empty›();

    |K:return data_[head_++ |% capacity]; |-
  }
  »»
\end{frame}

\begin{frame}{Now consider this}{}
  Same thing, but:
  \begin{itemize}
    \item no mutual exclusion
    \item only two threads:
      \begin{itemize}
        \item one only enqueues
        \item one only dequeues
      \end{itemize}
  \end{itemize}
\end{frame}

\begin{frame}[fragile]{Wait-free SRSW FIFO}{}
  ««
  |P‹|#include <array>›
  |P‹|#include <atomic>›

  |K:template <|K:typename Element, |T:int capacity>
  |K:class |D‹Wf_SRSW_FIFO›
  {
  |K‹public›:
  	|+|T:void enq(|T‹Element›);
    |T:Element deq(); |-

  |K‹private›:
  	|+|T‹std::array<Element, capacity>› data_;
    |T‹std::atomic<unsigned long>› head_{|L0}, tail_{|L0}; |-
  };
  »»
\end{frame}

\begin{frame}[fragile]{Wait-free SRSW enqueue}{}
  ««
  |K:template <|K:typename Element, |T:int capacity>
  |T:void |D‹Wf_SRSW_FIFO<Element, capacity>›::enq(|T:Element x)
  {
    	|+|K:if (tail_ - head_ == capacity) |K:throw |D‹fifo_full›();

    data_[tail_ |% capacity] = x;
    ++tail_; |-
  }
  »»
\end{frame}

\begin{frame}[fragile]{Wait-free SRSW deque}{}
  ««
  |K:template <|K:typename Element, |T:int capacity>
  |T:Element |D‹Wf_SRSW_FIFO<Element, capacity>›::deq()
  {
    	|+|K:if (tail_ == head_) |K:throw |D‹fifo_empty›();

    |T:Element result = data_[head_ |% capacity];
    ++head_;
    |K:return result; |-
  }
  »»
\end{frame}

\begin{frame}{What \emph{is} a concurrent queue?}{}
  \begin{itemize}
    \item Need a way to \alert{specify} a concurrent queue object
    \item Need a way to \alert{prove} that an algorithm implements the
      spec
  \end{itemize}
  \pause
  How do we specify objects?
\end{frame}

\begin{frame}{Object specification}{}
  In a concurrent setting:
  \begin{itemize}
    \item it gets the right answer (correctness, a safety property)
    \item it doesn’t get stuck (progress, a liveness property)
  \end{itemize}
  Let’s start with correctness.
\end{frame}

\begin{frame}{Sequential objects}{}
  Each object has:
  \begin{itemize}
    \item a \alert{state}:
      \begin{itemize}
        \item fields, usually
        \item FIFO example: the sequence of elements
      \end{itemize}
    \item a set of \alert{methods}:
      \begin{itemize}
        \item only way to access/manipulate the state
        \item FIFO example: «enq» and «deq» methods
      \end{itemize}
  \end{itemize}
\end{frame}

\begin{frame}{Sequential specification}{}
  \begin{itemize}
    \item If \alert{(precondition)}
      \begin{itemize}
        \item the object is in such-and-such a state
        \item before you call the method,
      \end{itemize}
    \item<2-> then \alert{(postcondition)}
      \begin{itemize}
        \item the method will return a particular value
        \item or throw a particular exception
      \end{itemize}
    \item<3-> and \alert{(postcondition)}
      \begin{itemize}
        \item the object will be in some specified state
        \item when the method returns.
      \end{itemize}
  \end{itemize}
\end{frame}

\begin{frame}{Example sequential specification: dequeue}
  \begin{itemize}
    \item Precondition:
      \begin{itemize}
        \item queue state is $x_1, x_2, \ldots, x_k$ for $k \ge 1$
      \end{itemize}
    \item Postcondition:
      \begin{itemize}
        \item returns $x_1$
      \end{itemize}
    \item Postcondition:
      \begin{itemize}
        \item queue state is $x_2, \ldots, x_k$
      \end{itemize}
  \end{itemize}
  \pause
  Easy!
\end{frame}

\begin{frame}[fragile]{Example sequential specification: dequeue}{}
  \begin{itemize}
    \item Precondition:
      \begin{itemize}
        \item queue is empty
      \end{itemize}
    \item Postcondition:
      \begin{itemize}
        \item throws «|D‹fifo_empty›» exception
      \end{itemize}
    \item Postcondition:
      \begin{itemize}
        \item state is unchanged
      \end{itemize}
  \end{itemize}
  Easy!
\end{frame}

\begin{frame}{Sequential specifications are awesome!}{}
  \begin{itemize}
    \item All method interactions captures by side-effects on state
    \item Each method described in isolation
    \item Can add new methods easily
  \end{itemize}
  \pause
  What about concurrent specification?
\end{frame}

\begin{frame}{Complication: methods take \only<2->{overlapping }time}{}
  \begin{itemize}
    \item Sequential: what is time? who cares?
    \item Concurrent: method call is \alert{interval}, not \alert{event}
  \end{itemize}
  \pause
  \begin{itemize}
    \item Sequential: invariants must hold \alert{between} calls
    \item Concurrent: overlapping means might \alert{never} be between
      calls
  \end{itemize}
\end{frame}

\begin{frame}{The Big Question}{}
  What does it \alert{mean} for a \emph{concurrent} object to be
  correct?
  \par\smallskip
  \pause
  Or, what \alert{is} a concurrent FIFO queue?
  \pause
  \begin{itemize}
    \item FIFO means stuff happens in order
    \item concurrent means time/order is kinda ambiguous
  \end{itemize}
\end{frame}

\begin{frame}[fragile]{Intuitively\ldots}{}
  ««
  |K:template <|K:typename Element, |T:int capacity>
  |T:void |D‹Lock_based_FIFO<Element, capacity>›::enq(|T:Element x) {
  	|+|color‹red›LockGuard guard(lock_);
    |K:if (tail_ - head_ == capacity) |K:throw |D‹fifo_full›();
    data_[tail_++ |% capacity] = x; |-
  }

  |K:template <|K:typename Element, |T:int capacity>
  |T:Element |D‹Lock_based_FIFO<Element, capacity>›::deq() {
  	|+|color‹red›LockGuard guard(lock_);
    |K:if (tail_ == head_) |K:throw |D‹fifo_empty›();
    |K:return data_[head_++ |% capacity]; |-
  }
  »»
  \pause
  Mutual exclusion means we can describe the behavior sequentially
\end{frame}

\begin{frame}{Linearizability}{}
  \begin{itemize}
    \item Each method “takes effect” “instantaneously” between invocation
      and response events
    \item Object is correct if this “sequential” behavior is correct
  \end{itemize}
  \pause
  Such a concurrent object is \emph{linearizable}
\end{frame}

\begin{frame}{Is linearizability really obout the object?}{}
  A linearizable object: all of its possible \alert{executions} are
  linearizable

  {\small (Linearizable execution examples on board)}
\end{frame}

\begin{frame}{Formal model of executions}{}
  Split method call into two events:
  \begin{center}
    \begin{tabular}{|l|l|l|}
      \hline
      Invocation & A q.enq(x) & Thread A calls q.enq(x) \\
      \hline
      Response & A q:void & Result is void \\
      \hline
    \end{tabular}
  \end{center}
\end{frame}

\def\A{\color{solarizedRed}}
\def\B{\color{solarizedBlue}}
\def\C{\color{solarizedGreen}}
\def\D{\color{solarizedViolet}}

\begin{frame}<1-4>{Definition: History}
  \uncover<2-3>{\alt<2>{Object projection:}{Thread projection:}}

  \[
    H\uncover<2-3>{|\alt<2>{\text{q}}{\text{\B B}}} =
    \begin{tabular}{l}
      \uncover<1-2,4>{\A A q.enq(3)} \\
      \uncover<1-2,4>{\A A q:void} \\
      \uncover<1,3-4>{\B B p.enq(4)} \\
      \uncover<1,3-4>{\B B p:void} \\
      \B B q.deq() \\
      \B B q:3 \\
    \end{tabular}
  \]
\end{frame}

\begin{frame}{\emph{Complete} subhistories}
  Remove pending invocations:

  \[
    \uncover<2>{\text{Complete}(}H\uncover<2>{)} =
    \begin{tabular}{l}
      \A A q.enq(3) \\
      \A A q:void \\
      \uncover<1>{\A A q.deq()} \\
      \B B p.enq(4) \\
      \B B p:void \\
      \B B q.deq() \\
      \B B q:3 \\
    \end{tabular}
  \]
\end{frame}

\begin{frame}{\emph{Sequential} subhistories}
  Responses immediately follow invocations (except possibly a final
  invocation):

  \[
    H =
    \begin{tabular}{l}
      \A A q.enq(3) \\
      \A A q:void \\
      \hline
      \B B p.enq(4) \\
      \B B p:void \\
      \hline
      \B B q.deq() \\
      \B B q:3 \\
      \hline
      \A A q.deq() \\
    \end{tabular}
  \]
\end{frame}

\begin{frame}{History \emph{well-formedness}}
  \[
    H = \begin{tabular}{l}
      \A A q.enq(3) \\
      \B B p.enq(4) \\
      \B B p:void \\
      \B B q:deq() \\
      \A A q:void \\
      \B B q:3 \\
    \end{tabular}
  \]
  \pause
  $H$ is well formed if its thread projections are sequential:
  \pause
  \begin{align*}
    H|{\text{\A A}} &= \begin{tabular}{l}
      \A A q.enq(3) \\
      \A A q:void \\
    \end{tabular}
    &
    H|{\text{\B B}} &= \begin{tabular}{l}
      \B B p.enq(4) \\
      \B B p:void \\
      \hline
      \B B q.deq() \\
      \B B q:3 \\
    \end{tabular}
  \end{align*}
\end{frame}

\begin{frame}{History equivalence}{}
  \begin{align*}
    H &= \begin{tabular}{l}
      \A A q.enq(3) \\
      \B B p.enq(4) \\
      \B B p:void \\
      \B B q:deq() \\
      \A A q:void \\
      \B B q:3 \\
    \end{tabular}
    &
    G &= \begin{tabular}{l}
      \A A q.enq(3) \\
      \A A q:void \\
      \B B p.enq(4) \\
      \B B p:void \\
      \B B q:deq() \\
      \B B q:3 \\
    \end{tabular}
  \end{align*}
  \pause
  $G \sim H$ iff threads see the same things:
  \begin{align*}
    H|\text{\A A} &= G|\text{\A A} \\
    H|\text{\B B} &= G|\text{\B B} \\
  \end{align*}
\end{frame}

\begin{frame}{Sequential specification}
  A \emph{sequential specification} describes a legal single-thread,
  single-object history

  \pause
  \bigskip
  A grammar for (\alert{unbounded}) FIFO histories:
  \begin{center}
    \begin{tabular}{r@{\quad::=\quad}l}
      $H$ & $H_\epsilon$ \\[8pt]
      $H_{x_1,\ldots,x_k}$ & \\
      $H_{x_1,\ldots,x_k}$ & q.enq($x$); q:void; $H_{x_1,\ldots,x_k,x}$ \\
      $H_{x_0,x_1,\ldots,x_k}$ & q.deq(); q:$x_0$; $H_{x_1,\ldots,x_k}$ \\
    \end{tabular}
  \end{center}
\end{frame}

\begin{frame}{\emph{Legal} histories}
  A sequential (multi-object, multi-thread) history $H$ is \emph{legal}
  if:
  \begin{center}
    For every object $x$, $H|x$ is in the sequential spec for
    $x$.
  \end{center}
\end{frame}

\begin{frame}{\emph{Precedence}}{}
  A method call $c$ \emph{precedes} a method call $d$ if $c$’s response
  comes before $d$'s invocation

  \pause
  Example:
  \[
    \begin{tabular}{l}
      \A A q.enq(3) \\
      \C B p.enq(4) \\
      \C B p:void \\
      \A A q:void \\
      \B B q.deq() \\
      \B B q:3 \\
    \end{tabular}
  \]

  \begin{itemize}
    \item Method call {\A A q.enq(3)} precedes method call
      {\B B q.deq()}
    \item Method call {\C A q.enq(4)} precedes method call
      {\B B q.deq()}
    \item Method call {\A A q.enq(3)} \alert{does not precede}
      method call {\C B q.enq(4)}
  \end{itemize}
\end{frame}

\begin{frame}{Properties of precedence}{}
  \begin{itemize}
    \item In general, it’s a partial order
    \item For a sequential history, it’s a total order
  \end{itemize}
  Have we seen this before?

  \pause
  Yes: Precedence is \emph{happens-before} $(\to)$ for method call
  intervals
\end{frame}

\begin{frame}{Linearizability, formally}
  History $H$ is \emph{linearizable} if it can be extended to complete
  history $G$ by
  \begin{itemize}
    \item appending responses to some pending invocations, and/or
    \item discarding the remaining pending invocations
  \end{itemize}
  such that there exists some legal sequential history $S \sim G$ where
  ${\to_H} \subseteq {\to_S}$
\end{frame}

\begin{frame}{Example}{}
  \begin{align*}
    H &= \begin{tabular}{l}
      \A A q.enq(3) \\
      \B B q.enq(4) \\
      \B B q:void \\
      \C B q.deq() \\
      \C B q:4 \\
      \D B q.enq(6) \\
      \\
    \end{tabular}
    &
    \pause
    G &= \begin{tabular}{l}
      \A A q.enq(3) \\
      \B B q.enq(4) \\
      \B B q:void \\
      \C B q.deq() \\
      \C B q:4 \\
      \\
      \A A q:void \\
    \end{tabular}
    &
    \pause
    S &= \begin{tabular}{l}
      \B B q.enq(4) \\
      \B B q:void \\
      \A A q.enq(3) \\
      \A A q:void \\
      \C B q.deq() \\
      \C B q:4 \\
      \\
    \end{tabular}
  \end{align*}
  \pause
  \begin{itemize}
    \item $S$ is legal and sequential
    \item $S \sim G$
    \item ${\to_H} \subseteq {\to_S}$
  \end{itemize}
\end{frame}

\begin{frame}{Composability theorem}
  History $H$ is linearizable if for every object $x$, $H|x$ is
  linearizable

  \pause\medskip
  This means we can reason about objects independently
\end{frame}

\begin{frame}{}{}
\vskip4pt
\parskip=4pt
\scriptsize
This work is licensed under a Creative Commons “Attribution-ShareAlike
3.0 Unported” license.
\par
These slides are derived from the companion slides for \emph{The Art of
Multiprocessor Programming,} by Maurice Herlihy and Nir Shavit. Its
original license reads:
\begin{quote}
  This work is licensed under a \emph{Creative Commons Attribution-ShareAlike
  2.5 License.}
\begin{itemize}
  \item \textbf{You are free}:
    \begin{itemize}
      \item\tiny \textbf{to Share} — to copy, distribute and transmit the work
        \item\tiny \textbf{to Remix} — to adapt the work
    \end{itemize}
  \item \textbf{Under the following conditions:}
    \begin{itemize}
      \item\tiny \textbf{Attribution.} You must attribute the work to “The Art of
        Multiprocessor Programming” (but not in any way that suggests
        that the authors of that work or this endorse you or your use of
        the work).
      \item\tiny \textbf{Share Alike.} If you alter, transform, or build upon this work,
        you may distribute the resulting work only under the same,
        similar or a compatible license.
    \end{itemize}
  \item For any reuse or distribution, you must make clear to others the
    license terms of this work. The best way to do this is with a link to
    \begin{itemize}
      \item\tiny \url{http://creativecommons.org/licenses/by-sa/3.0/}.
    \end{itemize}
  \item Any of the above conditions can be waived if you get permission from
    the copyright holder.
  \item Nothing in this license impairs or restricts the author’s moral
    rights.
\end{itemize}
\end{quote}
\end{frame}

\end{document}
